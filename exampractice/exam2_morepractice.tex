\documentclass{article}

\usepackage{amsfonts}
\usepackage{amsmath}
\usepackage{amssymb}
\usepackage{amsthm}
\usepackage{caption}
\usepackage{color}
\usepackage{enumerate}
\usepackage{fancyhdr}
\usepackage{hyperref}
\usepackage{graphicx}
\usepackage{latexsym}
\usepackage{listings}
\usepackage{mathrsfs}
\usepackage{natbib}
\usepackage[nottoc]{tocbibind}
\usepackage{url}

\providecommand{\all}{\ \forall \ }
\providecommand{\bs}{\backslash}
\providecommand{\e}{\varepsilon}
\providecommand{\E}{\ \exists \ }
\providecommand{\lm}[2]{\lim_{#1 \rightarrow #2}}
\providecommand{\m}[1]{\mathbb{#1}}
\providecommand{\nv}{{}^{-1}}
\providecommand{\ov}[1]{\overline{#1}}
\providecommand{\p}{\newpage}
\providecommand{\q}{$\quad$ \newline}
\providecommand{\rt}{\rightarrow}
\providecommand{\Rt}{\Rightarrow}
\providecommand{\vc}[1]{\boldsymbol{#1}}
\providecommand{\wh}[1]{\widehat{#1}}

%\renewcommand\bibname{References}
\renewcommand{\thesection}{Problem \arabic{section}}
\renewcommand{\thesubsection}{Part \alph{subsection}}
\numberwithin{equation}{section}

\fancyhead{}
\fancyfoot{}
\fancyhead[R]{\thepage}
\fancyhead[C]{Landau}

\hypersetup{
    colorlinks,
    citecolor=black,
    filecolor=black,
    linkcolor=black,
    urlcolor=blue
}

\definecolor{dkgreen}{rgb}{0,0.6,0}
\definecolor{gray}{rgb}{0.5,0.5,0.5}
\definecolor{mauve}{rgb}{0.58,0,0.82}

\lstset{ 
  language=C,                % the language of the code
  basicstyle=\Large,           % the size of the fonts that are used for the code
  numberstyle= \tiny \color{white},  % the style that is used for the line-numbers
  stepnumber=2,                   % the step between two line-numbers. 
  numbersep=5pt,                  % how far the line-numbers are from the code
  backgroundcolor=\color{white},      % choose the background color. You must add \usepackage{color}
  showspaces=false,               % show spaces adding particular underscores
  showstringspaces=false,         % underline spaces within strings
  showtabs=false,                 % show tabs within strings adding particular underscores
  frame=lrb,                   % adds a frame around the code
  rulecolor=\color{black},        % if not set, the frame-color may be changed on line-breaks within not-black text 
  tabsize=2,                      % sets default tabsize to 2 spaces
  captionpos=t,                   % sets the caption-position 
  breaklines=true,                % sets automatic line breaking
  breakatwhitespace=false,        % sets if automatic breaks should only happen at whitespace
  title=\lstname,                   % show the filename of files included with \lstinputlisting;
  keywordstyle=\color{blue},          % keyword style
  commentstyle=\color{gray},       % comment style
  stringstyle=\color{dkgreen},         % string literal style
  escapeinside={\%*}{*)},            % if you want to add LaTeX within your code
  morekeywords={*, ...},               % if you want to add more keywords to the set
  xleftmargin=0.053in, % left horizontal offset of caption box
  xrightmargin=-.03in % right horizontal offset of caption box
}

\DeclareCaptionFont{white}{\color{white}}
\DeclareCaptionFormat{listing}{\parbox{\textwidth}{\colorbox{gray}{\parbox{\textwidth}{#1#2#3}}\vskip-0.05in}}
\captionsetup[lstlisting]{format = listing, labelfont = white, textfont = white}
% For caption-free listings, comment out the 3 lines above and uncomment the 2 lines below.
% \captionsetup{labelformat = empty, labelsep = none}
% \lstset{frame = single}

\begin{document}

\begin{flushleft}

\begin{enumerate}[1. ]
\item Let $X$, $Y$, and $Z$ be random variables with expected values and standard deviations given below:

\begin{center}
\begin{tabular}{ccc}
& Expected Value & Standard Deviation \\ \hline
$X$ & 1.5 & 3.2 \\ 
$Y$ & 0 & 8.1 \\ 
$Z$ & 6 & 2.7 \\ 
\end{tabular}
\end{center}

Find:
\begin{itemize}
\item $E(8 + 2 X + Y + Z)$
\item $SD(8 + 2 X + Y + Z)$
\end{itemize}

{\color{red}

\begin{itemize}
\item The expected value of the linear combination is:
\begin{align*}
E(8 + 2 X + Y + Z) &= 8 + 2 E(X) + E(Y) + E(Z) \\
&= 8 + 2 \cdot 1.5 + 0 + 6 \\
&= 17
\end{align*}
\item Before computing the standard deviation, note:
\begin{align*}
Var(8 + 2 X + Y + Z) &= 2^2 Var(X) + Var(Y) + Var(Z)
\intertext{Remember that the standard deviation is the square root of the variance:}
[SD(8 + 2 X + Y + Z)]^2 &= 2^2 [SD(X)]^2 + [SD(Y)]^2 + [SD(Z)]^2 \\
SD(8 + 2 X + Y + Z) &= \sqrt{2^2 [SD(X)]^2 + [SD(Y)]^2 + [SD(Z)]^2} \\
&= \sqrt{2^2 [3.2]^2 + [8.1]^2 + [2.7]^2} \\
&\approx 10.671
\end{align*}
\end{itemize}
}


\item Let $X$ be the the number of crankshafts that fail in a given test of a certain type of vehicle ($X = 0, 1, 2$). Let $Y = 1$ if the clutch fails during that same test and $Y = 0$ otherwise. Consider the joint distribution of $X$ and $Y$:

\begin{center}
\begin{tabular}{cccc}
 $Y \backslash X$ & 0 & 1 & 2 \\ \hline
 0 & 0.35 & 0.1 & 0.05 \\
 1 & 0.2 & 0.25 & 0.05 \\
\end{tabular}
\end{center}

Find or answer the following:

\begin{itemize}
\item $P(X = 1$ and $ Y = 1)$
\item $P(X  = 0)$
\item $P(X > 0$ and $Y = 1)$
\item The marginal pmfs of $X$ and $Y$
\item Are $X$ and $Y$ independent? Why or why not?
\end{itemize}

{\color{red}

\begin{itemize}
\item $P(X = 1$ and $ Y = 1)$ = 0.25 from the table.
\item $P(X = 0)$:
\begin{align*}
P(X = 0) &= P(X = 0, Y = 0) + P(X = 0, Y = 1) \\
&= 0.35 + 0.2 \\
&= 0.55
\end{align*},
\item $P(X > 0$ and $Y = 1)$
\begin{align*}
P(X > 0, Y = 1) &= P(X = 1, Y = 1) + P(X = 2, Y = 1) \\
&= 0.25 + 0.05 \\
&= 0.3
\end{align*}
\item For the marginal pmf of $X$, take the row sums of the table: 

\begin{center}
\begin{tabular}{cccc}
$x$ & 0 & 1 & 2 \\ \hline
$f_X(x)$ & 0.55 & 0.35 & 0.1 
\end{tabular}
\end{center}

For the marginal pmf of $Y$, take the column sums of the table:

\begin{center}
\begin{tabular}{ccc}
$y$ & 0 & 1  \\ \hline
$f_Y(y)$ & 0.5 & 0.5
\end{tabular}
\end{center}
\item $X$ and $Y$ are independent random variables if and only if $P(X = x, Y = y) = P(X = x) \cdot P(Y = y)$ for all values $x$ and $y$. That is, the joint pmf must always be the product of the two marginals. However, in this case, $P(X = 1, Y = 1)$ = 0.25, while $P(X = 1) \cdot P(Y = 1) = f_X(1) \cdot f_Y(1) = 0.35 \cdot 0.5 = 0.175$. Therefore, $X$ and $Y$ are not independent.
\end{itemize}}

\end{enumerate}


\end{flushleft}
%\newpage 
%\nocite{*}
%\bibliographystyle{plainnat} 
%\bibliography{}
\end{document}